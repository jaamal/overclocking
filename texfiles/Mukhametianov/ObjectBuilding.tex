\subsection{Конструирование объекта}

Конструирование объекта происходит в тот момент, когда уже определено, экземпляр какого конкретно класса должен быть создан. Помимо конкретного класса также известен список
требуемых подстановок в конструктор в случае параметризованного запроса на создание. Список подстановок представляет из себя массив пар <<тип-экземпляр>>.
В конструировании можно выделить три основных этапа:
\begin{enumerate}
	\item выбор подходящего конструктора
	\item получение аргументов для выбранного конструктора
	\item вызов конструктора для создания объекта
\end{enumerate}

\subsubsection{Выбор подходящего конструктора}

Получить список всех конструкторов класса можно с помощью вызова метода getConstructors(). Подходящий конструктор будем искать среди всех следующим алгоритмом:
\begin{enumerate}
	\item Если конструкторов вообще не найдено "--- будет вызвано исключение
	\item Если конструктор найден и он ровно один, то он будет считаться подходящим
	\item Будут найдены все конструкторы, помеченные аннотацией

@ContainerConstructor
	\item Если этих конструкторов несколько "--- будет вызвано исключение
	\item Если этот конструктор ровно один "--- он будет считаться подходящим. Во всех остальных случаях ни один конструктор не проаннотирован как @ContainerConsturctor
	\item Если список требуемых подстановок не null и существует конструктор, подходящий под этот список, то он будет считаться подходящим
	\item Если конструктор с непустым списком аргументов ровно один, то он будет считаться подходящим
	\item Иначе будет вызвано исключение "--- невозможно однозначно определить, какой именно конструктор использовать для сборки объекта
\end{enumerate}

Определять, подходит ли конструктор под список требуемых подстановок будем с помощью следующего алгоритма:
\begin{enumerate}
	\item Помечаем все подстановки в списке неиспользованными
	\item Для каждого аргумента конструктора проверяем, является ли его тип предком типа элемента на соответствующей позиции в списке подстановок
	\item Если является, то помечаем подстановку как использованную
	\item Иначе проверяем, сможем ли мы получить экземпляр данного типа из контейнера, если не сможем "--- конструктор не подходит
	\item После всех этих действий для всех аргументов проверяем, что все подстановки использованы. Если все "--- конструктор подходит, иначе нет
\end{enumerate}

\subsubsection{Получение аргументов для выбранного конструктора}

Конструктор выбран, теперь необходимо построить список объектов для передачи их в качестве аргументов в этот конструктор для последующего создания объекта.
На этом этапе будет активно использоваться список требуемых подстановок, если он не пуст. Используется следующий алгоритм:

\begin{enumerate}
	\item Помечаем все подстановки в списке неиспользованными
	\item Просматриваем список типов аргументов выбранного конструктора
	\item Если существует неиспользованная подстановка, тип которой является наследником типа текущего аргумента, то
		\begin{enumerate}
			\item Помечаем данную подстановку как использованную
			\item Подставляем в качестве очередного аргумента объект из данной подстановки
		\end{enumerate}
	\item В противном случае подставляем значение, которое вернет get-запрос по данному типу к контейнеру
\end{enumerate}

\subsubsection{Вызов конструктора для создания объекта}

Мы определились с выбором конструктора и построили массив аргументов для передачи в этот конструктор. Осталось вызвать у выбранного конструктора метод newInstance(parameters), и мы получим
необходимый объект.