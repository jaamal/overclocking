\subsection{Фабрики}

Бывают ситуации, когда необходимо создать несколько объектов одного типа, которые отличаются лишь аргументами, которые следует передать в конструктор.
С помощью контейнера это можно сделать вручную с помощью множества вызовов метода create с нужными подстановками. Например, если у нас есть класс Sample с единственным
конструктором, принимающим на вход два аргумента типа int, то создать несколько экземпляров с разными аргументами можно следующим кодом:

\begin{lstlisting}[language=Java,caption={Пример создания множества экземпляров с разными параметрами}]
container.create(Sample.class, new AbstractionInstancePair(int.class, 0), new AbstractionInstancePair(int.class, 1));
container.create(Sample.class, new AbstractionInstancePair(int.class, 2), new AbstractionInstancePair(int.class, 3));
container.create(Sample.class, new AbstractionInstancePair(int.class, 4), new AbstractionInstancePair(int.class, 5));
\end{lstlisting}

Был реализован механизм, позволяющий избежать использования контейнера напрямую (то есть непосредственно вызова метода create).
Суть его следующая: вместо того, чтобы создавать объекты прямо конструктором, переложить эту обязанность на контейнерные фабрики. Контейнерная фабрика "--- интерфейс, который имеет методы
create с нужными аргументами, возвращающие значения нужного типа, реализация которого генерируется самим контейнером. 
Например, для указанного выше примера должен быть создан следующий интерфейс:

\begin{lstlisting}[language=Java,caption={Пример фабрики}]
@ContainerFactory
public interface ISampleFactory
{
	Sample create(int a, int b);
}
\end{lstlisting}

Аннотация @ContainerFactory указывает контейнеру, что данная абстракция является контейнерной фабрикой, и необходимо сгенерировать ее реализацию. Например, для нашего примера будет
сгенерирован следующий класс:

\begin{lstlisting}[language=Java,caption={Реализация интерфейса ISampleFactory, которая будет сгенерирована контейнером}]
public class ISampleFactoryGeneratedImplementation
{
	private IContainer container;
	
	public ISampleFactoryGeneratedImplementation(IContainer container)
	{
		this.container = container;
	}
	
	public Sample create(int a, int b)
	{
		return container.create(Sample.class, new AbstractionInstancePair[]{new AbstractionInstancePair(int.class, a), new AbstractionInstancePair(int.class, b)});
	}
}
\end{lstlisting}

И теперь код, приведенный в начале главы, преобразится в следующий:

\begin{lstlisting}[language=Java,caption={Код с использованием контейнерных фабрик}]
factory.create(0, 1);
factory.create(2, 3);
factory.create(4, 5);
\end{lstlisting}

Видно, что теперь создавать объекты стало гораздо удобнее, и для этого надо всего лишь создать интерфейс и в нужном месте принять этот интерфейс в конструктор,
контейнер по аннотации @ContainerFactory поймет, что этот интерфейс является фабрикой и сам сгенерирует его реализацию.

Более формально процесс генерации реализации можно описать следующим образом:

\begin{enumerate}
	\item Создать поле, в котором будет лежать контейнер
	\item Создать конструктор, принимающий в качестве единственного аргумента контейнер
	\item Просмотреть все методы интерфейса, имеющие имя create, и создать реализацию каждого такого метода
\end{enumerate}

Для метода T create(T1 t1, T2 t2, ..., Tn tn) будет сгенерирован следующий код:

\begin{lstlisting}[language=Java,caption={Общий вид кода генерируемого метода}]
public T create(T1 t1, T2 t2, ..., Tn tn)
{
	return container.create(T.class, new AbstractionInstancePair[]{
			new AbstractionInstancePair(T1.class, t1),
			new AbstractionInstancePair(T2.class, t2),
			...,
			new AbstractionInstancePair(Tn.class, tn)});
}
\end{lstlisting}

Генерация кода реализована с помощью BCEL, которая позволяет генерировать объекты типа JavaClass и сериализовывать их в массив байт (непосредственно получать байткод). 
Полученный байткод загружается простейшим загрузчиком классов, после чего мы можем использовать полученную реализацию в коде.

Для создания класса используется ClassGen "--- генератор классов. Его основные используемые методы, использованные в работе, следующие:

\begin{itemize}
	\item addField "--- добавляет поле, переданное в качестве аргумента
	\item addMethod "--- добавляет метод, переданный в качестве аргумента
	\item getJavaClass "--- возвращает объект типа JavaClass, соответствующий сгенерированному классу
\end{itemize}

Поле представляет из себя класс Field, которое получается с помощью класса-генератора полей FieldGen. Ему указывается имя поля, модификаторы доступа, тип, а также 
другие аргументы.

Метод представляет из себя класс Method, который получается с помощью класса-генератора методов MethodGen.
Генератору методов указываются модификаторы доступа, тип возвращаемого значения, типы и имена аргументов, имя метода, имя класса, которому будет принадлежать метод, 
список инструкций и другие аргументы.

Список инструкций "--- список кодов команд, которые являются внутренним представлением исходного кода в JVM. По сути это некоторый аналог ассемблера для языка Java.
Для построения списков инструкций есть достаточно удобные встроенные в BCEL инструменты. Детали реализации можно посмотреть в исходном коде проекта.
