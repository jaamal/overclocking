\section{Описание реализованного контейнера}

Опишем реализованный контейнер в общих чертах.

В момент создания контейнер прочитает скомпилированные классы из указанных папок и построит на этих классах граф реализации; эти процессы описаны в параграфе 4.

Возможности точечной конфигурации описаны в параграфе 5.

При запросах на получение или создание вызываются методы, отвечающие за конструирование объекта. Алгоритмы конструирования описаны в параграфе 7. 

Приложение, написанное на Java, обычно состоит из множества модулей, каждый из которых должен быть загружен определенным загрузчиком классов, более подробно этот вопрос
освещен в параграфе 6.

Дополнительные возможности контейнера "--- протоколирование процесса построения объекта и автоматическая генерация классов-фабрик "--- описаны в параграфах 8 и 9.

\input Initialization

\input Configuration

\input ClassLoading

\input ObjectBuilding

\input Logging

\input Factories