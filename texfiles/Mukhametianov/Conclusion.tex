\section{Заключение}

В данной работе представлено описание реализованного DI-контейнера для языка Java, обладающего следующими отличительными особенностями:

\begin{itemize}
	\item минимальность конфигурации "--- пользователю не требуется явно перечислять все классы, используемые в приложении, и настраивать способ их получения: в большинстве случаев
	контейнер сам принимает соответствующее решение
	\item возможность автоматической генерации классов-фабрик
\end{itemize}

Полученный контейнер является законченной библиотекой и может быть внедрен как в новые проекты, так и в уже существующие. 

При текущей реализации граф наследования не изменяется в ходе работы приложения, в частности, туда невозможно добавить новые классы. 
Реализация возможности динамичности графа зависимостей добавит гибкости контейнеру и позволит расширить круг решаемых им задач.

Автор выражает благодарность научному руководителю Бурмистрову Ивану Сергеевичу за ценные советы по качеству кода и оформлению работы, 
Хворосту Алексею Александровичу за возможность внедрения контейнера в реальный сервис и ценные замечания по поводу работы,
а также Клепинину Александру Владимировичу за ценные консультации по тонкостям языка Java и ценные советы по оформлению работы.