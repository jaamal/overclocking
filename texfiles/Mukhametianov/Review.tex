\documentclass[12pt]{article}
\usepackage{url}
\usepackage[utf8]{inputenc}
\usepackage[english,russian]{babel}
\usepackage{amsmath}
\usepackage{amsthm}
\usepackage{amsfonts}
\usepackage{amssymb}
\usepackage{gastex}
\usepackage[dvips]{graphicx}
\usepackage[ruled,section]{algorithm}
\usepackage[noend]{algorithmic}
\usepackage[all]{xy}

\usepackage{pgf}
\usepackage{tikz}

\usepackage{setspace}
\setstretch{1.15}

\usepackage[left=2cm,right=2cm,top=2cm,bottom=2cm,bindingoffset=0cm]{geometry}
\pagestyle{empty}

\begin{document}

\begin{center}
\small{\textbf{Отзыв научного руководителя \linebreak
на бакалаврскую работу Д.И.Мухаметьянова}} \linebreak
\large{\textbf{<<Реализация DI-контейнера с минимальной конфигурацией для языка Java>>}}
\end{center}

Современная индустрия разработки программного обеспечения сталкивается с рядом сложных задач.
Одна из таких задач~-- быстрая реакция на часто меняющиеся требования и обстоятельства внешнего мира в условиях проекта
с большим количеством исходного кода. Для обеспечения требуемой скорости реакции архитектура проекта должна быть 
гибкой, чтобы вносить изменения и расширять функциональность было достаточно просто. 
Для объектно-ориентированных языков программирования одним из способов поддержания должного уровня гибкости является использование Dependency Injection контейнера:
инструмента, обеспечивающего разрешение зависимостей в проекте и, как следствие, существенно сокращающего количество ручной работы. 
Данный подход на текущий момент считается одним из самых перспективных и нашел широкое применение на практике.

В работе Д.И.Мухаметьянова приведен обзор современного состояния Dependency Injection контейнеров для языка Java, перечислены их преимущества и недостатки.
В частности, отмечен общий недостаток у большинства популярных контейнеров~-- необходимость вручную перечислять все необходимые классы. 
В процессе решения задачи был реализован собственный Dependency Injection контейнер, который лишен данного недостатка. 
Как следствие, полученный контейнер может быть внедрен в проекты любого масштаба с минимальными усилиями.

Результат рассматриваемой работы имеет практическую важность и внедрен в исследовательский веб-сервис. Решение подобного рода задач требует высокого уровня
квалификации разработчика. Считаю, что Д.И.Мухаметьянов своим решением доказал, что обладает необходимыми знаниями и навыками. 
Также хочу отметить его возрастающую самостоятельность как исследователя. Считаю, что работа заслуживает оценки <<отлично>>, а Д.И.Мухаметьянову может быть присуждена
степень бакалавра.

\begin{flushleft}
Научный руководитель \linebreak
аспирант кафедры алгебры и дискретной математики \linebreak
института математики и компьютерных наук УрФУ \linebreak
\end{flushleft}

\begin{flushright}
И.С.Бурмистров
\end{flushright}

\begin{flushleft}
19 июня 2013 года
\end{flushleft} 

\end{document}