\subsection{Google Guice}

\subsubsection{Способы конфигурирования}

Google Guice "--- контейнер, конфигурирование которого производится с помощью классов-конфигураторов и аннотаций. Создается контейнер с помощью следующего кода:

\begin{lstlisting}[language=Java,caption={Создание контейнера}]
Injector injector = Guice.createInjector(new Module());
\end{lstlisting}

Класс Module должен наследоваться от класса AbstractModule, который требует реализации одного метода "--- configure. В этом методе и будет описана конфигурация.
В этом месте нужно явно прописать, какой класс будет реализовывать конкретную абстракцию, или какой конкретно объект вернуть при запросе, например:

\begin{lstlisting}[language=Java,caption={Примеры конфигурирования в наследнике AbstractModule}]
@Override
protected void configure()
{
	bind(IAuthenticatorsProvider.class).to(AuthenticatorsProvider.class);
	bind(IGmailAuthenticator.class).toInstance(new GmailAuthenticator());
}
\end{lstlisting}

Для внедрения зависимости используется аннотация @Inject.

В отличие от Spring IoC Container, Google Guice не позволяет столь же просто сделать инъекцию в конструктор с аргументом типа массив, но позволяет сделать это с аргументом типа Set.
Для этого используется класс Multibinder.

Реализация нашего примера с использованием данного контейнера имеет следующие ключевые моменты:

\begin{lstlisting}[language=Java,caption={Конфигурирование}]
import com.google.inject.AbstractModule;
import com.google.inject.multibindings.Multibinder;

public class Module extends AbstractModule
{
    @Override
    protected void configure()
    {
        bind(IAuthenticatorsProvider.class).to(AuthenticatorsProvider.class);
        Multibinder<IAuthenticator> multibinder = Multibinder.newSetBinder(binder(), IAuthenticator.class);
        multibinder.addBinding().to(FacebookAuthenticator.class);
        multibinder.addBinding().to(GmailAuthenticator.class);
        multibinder.addBinding().to(LocalAuthenticator.class);
    }
}
\end{lstlisting}

\begin{lstlisting}[language=Java,caption={Конструктор класса AuthenticatorsProvider}]
@Inject
public AuthenticatorsProvider(Set<IAuthenticator> authenticators)
{
    this.authenticators = authenticators.toArray(new IAuthenticator[authenticators.size()]);
}
\end{lstlisting}

\begin{lstlisting}[language=Java,caption={Место получения экземпляра}]
IAuthenticatorsProvider provider = injector.getInstance(IAuthenticatorsProvider.class);
\end{lstlisting}

\subsubsection{Плюсы и минусы}

Google Guice, в отличие от Spring IoC Container, сохраняет тип, соответственно не придется каждый раз явно использовать приведение типа при запросе getInstance к контейнеру.

Однако проблема с необходимостью описания всех классов в конфигурации никуда не делась, просто вся рутина переместилась из XML-файла в класс-конфигуратор.
Более того, при внедрении контейнера придется прописывать @Inject во всех местах, где потребуется вмешательство контейнера, то есть придется вносить достаточно 
существенные изменения в изначальный код.