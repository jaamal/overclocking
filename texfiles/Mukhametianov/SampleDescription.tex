\subsection{Применение принципа инверсии зависимостей без использования DI-контейнеров}

Рассмотрим следующий пример. Есть сервис, поддерживающий несколько способов аутентификации: 
\begin{itemize}
	\item аутентификация с использованием сторонних сервисов
	\item аутентификация с использованием собственного механизма
\end{itemize}
Аутентификация будет производиться через интерфейс IAuthenticator, который описан в листинге 1:

\begin{lstlisting}[language=Java,caption={IAuthenticator}]
public interface IAuthenticator
{
    String authenticate(String login, String password);
    String getName();
}
\end{lstlisting}

Метод authenticate принимает на вход логин и пароль пользователя и возвращает идентификатор сессии или null, если аутентификация не удалась, 
метод getName возвращает имя способа аутентификации, который реализован данным классом. Необходимо реализовать интерфейс, описанный в листинге 2:

\begin{lstlisting}[language=Java,caption={IAuthenticatorsProvider}]
public interface IAuthenticatorsProvider
{
    IAuthenticator getAuthenticator(String name);
}
\end{lstlisting}

Метод getAuthenticator должен принять на вход одну строку name и вернуть реализацию способа аутентификации с таким именем или null, если такого нет.

Рассмотрим реализацию данной задачи без использования DI контейнеров.
В листинге 3 приведен код класса AuthenticatorsProvider, реализующего интерфейс IAuthenticatorsProvider.
Листинг 4 демонстрирует, как нужно создавать экземпляр этого класса без использования контейнера. Здесь LocalAuthenticator "--- класс, отвечающий за аутентификацию с использованием
собственного механизма, а GmailAuthenticator и FacebookAuthenticator "--- классы, отвечающие за аутентификацию с помощью сторонних сервисов.


\begin{lstlisting}[language=Java,caption={Реализация интерфейса IAuthenticatorsProvider}]
public class AuthenticatorsProvider implements IAuthenticatorsProvider
{
    IAuthenticator[] authenticators;

    public AuthenticatorsProvider(IAuthenticator[] authenticators)
    {
        this.authenticators = authenticators;
    }

    public IAuthenticator getAuthenticator(String authenticatorName)
    {
        for(IAuthenticator authenticator : authenticators)
            if(authenticator.getName().equals(authenticatorName))
                return authenticator;
        return null;
    }
}
\end{lstlisting}

\begin{lstlisting}[language=Java,caption={Место получения (в данном случае создания) экземпляра класса AuthenticatorsProvider}]
IAuthenticatorsProvider provider = new AuthenticatorsProvider(new IAuthenticator[]{new LocalAuthenticator(), new FacebookAuthenticator(), new GmailAuthenticator()});
\end{lstlisting}

В листинге 4 показан кусок достаточно рутинного кода. В действительности, когда классов на порядок больше, количество рутины соответственно возрастет.
Dependency Injection контейнеры берут на себя ответственность за создание объектов, тем самым избавляя разработчика от ручного труда.