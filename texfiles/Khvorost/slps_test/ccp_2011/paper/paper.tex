\documentclass[10pt, conference, compsocconf]{IEEEtran}
\IEEEoverridecommandlockouts

\usepackage{cite}
\usepackage{pgf}
\usepackage{amsmath,amssymb}
\usepackage{amsthm}
\usepackage{url}
\usepackage{tikz}
\usepackage{common}

% correct bad hyphenation here
\hyphenation{op-tical net-works semi-conduc-tor}

\begin{document}

\title{Straight-line Programs: A Practical Test\thanks{The authors acknowledge support
from the Ministry for Education and Science of Russia, grant 2.1.1/13995, and
from the Russian Foundation for Basic Research, grant 10-01-00524.}}

\author{\IEEEauthorblockN{Ivan Burmistrov}
\IEEEauthorblockA{Faculty of Mathematics and Mechanics \\
Ural State University\\
Ekaterinburg, Russia\\
burmistrov.ivan@gmail.com}
\and
\IEEEauthorblockN{Lesha Khvorost}
\IEEEauthorblockA{Faculty of Mathematics and Mechanics \\
Ural State University\\
Ekaterinburg, Russia\\
jaamal@mail.ru}
}

\maketitle

\begin{abstract}
We present an improvement of Rytter's algorithm that constructs a straight-line program for a given text and show that
the improved algorithm is optimal in the worst case with respect to the number of AVL-tree rotations. Also we compare
Rytter's and ours algorithms on various data sets and provide a comparative analysis of compression ratio achieved by
these algorithms, by LZ77 and by LZW.
\end{abstract}

\begin{IEEEkeywords}
straight-line program; LZ-compression; AVL-tree;
\end{IEEEkeywords}

\IEEEpeerreviewmaketitle

\section{Introduction}

Nowadays searching algorithms on huge data sets attract much attention. To reduce the input size one needs algorithms
that can work directly with a compressed representation of input data.

Various compressed representations of strings are known: straight-line programs (SLPs)~\cite{SLPConstruction},
collage-systems \cite{collages}, string representations using antidictionaries \cite{antidictionaries}, etc. Nowadays
text compression based on context-free grammars such as SLPs has become a popular research direction. The reason for
this is not only that grammars provide well-structured compression but also that the SLP-based compression is,
in a sense, polynomially equivalent to the compression achieved by the Lempel-Ziv algorithm that is widely used in
practice. It means that, given a text $S$, there is a polynomial relation between the size of an SLP that
derives $S$ and the size of the dictionary stored by the Lempel-Ziv algorithm, see \cite{SLPConstruction}. It should
also be noted that classical LZ78 \cite{LZ78} and LZW \cite{LZW} algorithms can be considered as special cases of grammar
compression. (At the same time other compression algorithms from the Lempel-Ziv family---such as LZ77 \cite{LZ77} and
run-length compression---do not fit directly into grammar compression model.)

Using the fact that SLPs are nicely structured, several researchers keep developing analogues of classical string
algorithms that (at least theoretically) perform quite well on SLP-compressed representations: \textbf{Pattern matching}
\cite{PM_and_HD}, \textbf{Longest common substring} \cite{LCSubstring}, \textbf{Computing all palindromes} \cite{LCSubstring}, some
versions of \textbf{Longest common subsequence} \cite{LCS_P}. At the same time, constants hidden in big-O notation for
algorithms on SLPs are often very big. Also the aforementioned polynomial relation between the size of an SLP for a
given text and the size of the LZ77-dictionary for the same text does not yet guarantee that SLPs provide good
compression ratio in practice. Thus, a major questions is whether or not there exist SLP-based compression models suitable
to practical usage? This question splits into two sub-questions addressed in the present paper: How difficult is it to
compress data to an SLP-representation? How large compression ratio do SLPs provide as compared to
classic algorithms used in practice?

Let us describe in more detail the content of the paper and its structure. Section 2 gathers some preliminaries about
SLPs. In Section 3 we present an improved version of Rytter's algorithm~\cite{SLPConstruction} for constructing
an SLP-presentation of a given text. In Section 4 we compare the improved algorithm vs. the original algorithm
from~\cite{SLPConstruction} and also present results of a comparison of compression ratio between the two SLP-based
algorithms and LZ77. In Section 5 we summarize our results.

\section{Preliminaries}

We consider strings of characters from a fixed finite alphabet $\Sigma$. The \emph{length} of a string
$S$ is the number of its characters and is denoted by $|S|$. The \emph{concatenation} of strings $S_1$ and $S_2$
is denoted by $S_1 \cdot S_2$. A \emph{position} in a string $S$ is a point between consecutive characters. We number
positions from left to right by $1,2,\dots,|S|-1$. It is convenient to consider also the position 0 preceding the text
and the position $|S|$ following it. For a string $S$ and an integer $0 \leq i \leq |S|$ we define $S[i]$ as character
between positions $i$ and $i+1$ of $S$. For example $S[0]$ is a first character of $S$. A \emph{substring} of $S$
starting at a position $\ell$ and ending at a position $r$, $0\leq \ell < r \leq |S|$, is denoted by $S[\ell \dots r]$ (in other
words $S[\ell \dots r] = S[\ell] \cdot S[\ell + 1] \cdot \mbox{\dots} \cdot S[r-1]$).

A \emph{straight-line program} (SLP) $\slp{S}$ is a sequence of assignments of
the form: $$\slp{S}_1 = expr_1,\ \slp{S}_2 = expr_2, \dots, \slp{S}_n =
expr_n,$$ where $\slp{S}_i$ are \emph{rules} and $expr_i$ are expressions of
the form:
\begin{itemize}
\item $expr_i$ is a character of $\Sigma$ (we call such rules \emph{terminal}), or
\item $expr_i = \slp{S}_\ell\cdot \slp{S}_r \ (\ell, r < i)$ (we call such rules \emph{nonterminal}).
\end{itemize}

Thus, an SLP is a context-free grammar in Chomsky normal form. Obviously every SLP generates exactly one string over
$\Sigma^+$. This string is referred to as the \emph{text} generated by an SLP. For a grammar $\slp{S}$ generating a text
$S$, we define \emph{parse-tree} of $S$ as a derivation tree of $S$ in $\slp{S}$. We identify terminal symbols with
their parents in this tree; after this identification every internal node has exactly two sons. We accept the
following conventions in the paper: every SLP is denoted by a capital blackboard bold letter, for example, $\slp{S}$.
Every rule of this SLP (and every internal node in its parse-tree) is denoted by the same letter with indices, for
example, $\slp{S}_1,\slp{S}_2,\dots$. The \emph{size} of an SLP $\slp{S}$ is the number of its rules and is denoted by
$|\slp{S}|$. The \emph{concatenation} of SLPs $\slp{S}_1$ and $\slp{S}_2$ is an SLP that derives
$S_1 \cdot S_2$ and denoted by $\slp{S}_1 \cdot \slp{S}_2$.

The \emph{height} of a node in a binary tree is defined as follows. The height
of terminal node (leaf) is equal to 0 by definition. The height of a
nonterminal node is equal to 1 + the maximum of the heights of its children. An
\emph{AVL-tree} $\slp{T}$ is a binary tree such as for every non-terminal node
the heights of its children can differ at most by 1. In the paper we consider
only SLPs whose parse-trees are AVL-trees. We denote the height of the rule
$\slp{S}_i$ by $h(\slp{S}_i)$.

\section{Algorithms for constructing SLPs}

\subsection{Rytter's algorithm and its bottleneck}

The problem of constructing a minimal size grammar generating a given text is
known to be NP-hard. Hence we should look for polynomial-time approximation
algorithms. One of the key approaches of such algorithms is to construct a
factorization of a given text and to build some binary search tree using the
factorization. It is obvious that the size of the resulting grammar depends on
factorization. In \cite{SLPConstruction} Rytter considered factorizations
defined by the LZ77 encoding algorithm.

\begin{definition}
The LZ-factorization of a text $S$ is given by a decomposition: $S = F_1 \cdot F_2 \cdot \mbox{\dots} \cdot F_k$, where
$F_1 = S[0]$ and $F_i$ is the longest prefix of $S\substr{|F_1\cdot \mbox{\dots} \cdot F_{i-1}|}{|S|}$ which occurs as
substring in $F_1 \cdot \mbox{\dots} \cdot F_{i-1}$ or $S[|F_1\cdot \mbox{\dots} \cdot F_{i-1}|]$ in case this prefix is
empty.
\end{definition}

The following theorem gives an estimation of the size of an SLP constructed on
the basis on a LZ-factorization.

\begin{thm}[\rm{\cite{SLPConstruction}}]
Given a string $S$ of length $n$ and its LZ-factorization of length $k$, one
can construct an SLP for $S$ of size $O(k \log n)$ in time $O(k \log n)$.
\end{thm}

The proof of the above theorem contains an algorithm for constructing SLPs. We
remind here some key ideas of the algorithm as they are important for the
further discussion. For efficient concatenation of grammars the algorithm uses
the following lemma:

\begin{lem}[{\rm\cite{SLPConstruction}}]
Assume $\slp{S}_1, \slp{S}_2$ are two nonterminals of AVL-balanced grammars.
Then we can construct in $O(|h(\slp{S}_1) - h(\slp{S}_2)|)$ time an
AVL-balanced grammar $\slp{S} = \slp{S}_1 \cdot \slp{S}_2$ that derives the
text $S_1\cdot S_2$ by adding only $O(|h(\slp{S}_1) - h(\slp{S}_2)|)$
nonterminals.
\end{lem}


\noindent \textsc{Problem:} \textbf{SLP Construction}

\noindent \textsc{Input:} a string $S$ and its LZ-factorization $F_1, F_2,
\dots, F_k$.

\noindent \textsc{Output:} an SLP $\slp{S}$ that derives $S$.

\noindent \textsc{Algorithm:} The algorithm constructs an SLP by induction on
$k$.

\textbf{Base:} Initially $\slp{S}$ is equal to the terminal rule that derives
$S[0]$.

\textbf{Step:} Let $i > 1$ be an integer and an SLP $\slp{S}$ that derives
$F_1\cdot F_2 \cdots F_i$ has already been constructed. Since a
LZ-factorization of $S$ is fixed, an occurrence of $F_{i+1}$ in $F_1\cdot F_2
\cdots F_i$ is known. From this, the algorithm extracts rules $\slp{S}_1,
\dots, \slp{S}_\ell$ such that $F_{i+1} = S_1 \cdot S_2 \cdots S_\ell$. Since
$\slp{S}$ is balanced, we have $\ell = O(\log |S|)$. Then, using Lemma III.2,
the algorithm concatenates rules in some specific order (see
\cite{SLPConstruction} for details) and sets the next value of $\slp{S}$ to be
equal to the result of concatenating the previous value of $\slp{S}$ with
$\slp{S}_1 \cdots \slp{S}_\ell$.

\begin{figure}[th]
\AVLrotations
\end{figure}

It is well-known that adding a new rule to an AVL-tree is quite a complex
operation. Every adding operation generates a sequence of rotations of the
tree. There are two types of rotations symbolically presented in Figure 1, see
\cite{SLPConstruction} for more details. Every rotation may generate at most
three new rules. Also every rotation may generate three unused rules. There are
two possible directions for an optimization of the algorithm: to construct more
compact grammar and to minimize the number of queries to AVL-trees. Minimizing
of the number of queries to AVL-trees becomes important when the size of input
text becomes huge and we cannot store an AVL-tree in the memory. Formally it
means that costs of a query to an AVL-tree are greater than costs of
calculations in memory.

The following example illustrates a bottleneck of the algorithm
from~\cite{SLPConstruction}:

\noindent \textbf{Example 1:} Let $n$ be an integer and $S$ be a text of length
$2^n$. Suppose the algorithm has already built an SLP $\slp{S}$ that derives
$S\substr{0}{2^{n-1}}$. Let $F_1 \cdot F_2 \cdots F_{n-1}$ be LZ-factorization
of $S\substr{2^{n-1}}{2^n}$ such that $|F_i| = 2^{n-i-1}$, where $i \in
\overline{1\dots n-1}$.

Let us estimate the number of rotations that may be generated in the course of
the sequence of concatenations $(((\slp{S} \cdot \slp{F}_1)\cdot
\slp{F}_2)\dots) \cdot \slp{F}_{n-1}$ in the worst case. In order to get an
upper bound, we suppose that the resulting tree grows after every
concatenation. So we get the following estimation: \begin{multline*}
\sum_{i=1}^{n-1} (\log{2^{n-1}} + (i-1) - \log{2^{n-i-1}}) = \\
\frac{(n-1)(n-2)}{2}(1 + \log{2}) - (n-2) = \Theta(n^2).
\end{multline*}

Notice that if the algorithm could choose another order of concatenations, then
the number of rotations generated by the algorithm could be essentially
smaller, namely,close to $O(n)$. For instance, in the case of concatenating in
the reverse order $\slp{S} \cdot (\slp{F}_1 \dots (\slp{F}_{n-2} \cdot
\slp{F}_{n-1}))$, we get the following estimation:
$$\sum_{i=1}^{n-1} (\log{2^{n-i}} - \log{2^{n-i-1}}) = (n-1)\log{2} =
\Theta(n).$$

Our next example shows that several factors can be processed together if they
occur in a single SLP.

\noindent \textbf{Example 2:} Let $n > 0$ be an integer and $S = b\cdot
a^{2^{n-1}}\cdot b \cdot a^{2^{n-2}} \cdots b \cdot a$. So the length of $S$ is
equal to $2^n + n - 2$. Consider the LZ-factorization of $S$: $$ b \cdot a
\cdot a \cdot a^2 \cdot a^4 \cdots a^{2^{n-2} - 1} \cdot ba^{2^{n-2}} \cdot
ba^{2^{n-3}} \cdots ba.$$ Let $\slp{S}_1$ be an SLP that derives $b \cdot
a^{2^{n-1}}$. It is obvious that all other factors starting with $b \cdot
a^{2^{n-2}}$ occur in $S_1$, therefore the algorithm may process them together.
So the algorithm may construct SLPs $\slp{S}_2$ that derives $b \cdot
a^{2^{n-2}}$, $\slp{S}_3$ that derives $b \cdot a^{2^{n-3}}$, etc. Finally the
algorithm concatenates the SLPs in the following order:
$\slp{S}_1\cdot(\dots(\slp{S}_{n-3}\cdot(\slp{S}_{n-2} \cdot \slp{S}_{n-1})))$.

The main ideas of our improved  algorithm are to process several factors
together and to concatenate each group of factors choosing an optimal order.
The intuition behind the algorithm is very simple: if the algorithm has already
constructed a huge SLP, then most factors occur in the text generated by this
SLP and may be processed together.

\subsection{An improved algorithm}

\noindent \textsc{Input:} a text $S$ and its LZ-factorization $F_1, F_2, \dots,
F_k$.

\noindent \textsc{Output:} an SLP $\slp{S}$ that derives $S$.

\noindent \textsc{Algorithm:} The algorithm constructs an SLP by induction on
$k$.

\textbf{Base:} Initially $\slp{S}$ is equal to the terminal rule that derives
$S[0]$.

\textbf{Step:} Let $i$ be an integer and $\slp{S}$ be an SLP that derives
$F_1\cdot F_2 \cdots F_i$. Let $\ell \geq 1$ be a maximal integer such that
$F_{i+\ell}$ occurs in $F_1 \cdots F_i$. Since the LZ-factorization of $S$ is
fixed, the value of $\ell$ can be found by using a linear time search on
factors. Let $\slp{F}_1$ be an SLP that derives $F_{i+1}$, let $\slp{F}_2$ be
an SLP that derives $F_{i+2}$, etc.

Initially the algorithm concatenates $\slp{F}_{i+1}, \dots, \slp{F}_{i+k}$
using dynamic programming. Let $\varphi(p, q)$ be the function that calculates
an upper bound of the number of rotations of a grammar tree that are performed
during the process of concatenation of $\slp{F}_p, \slp{F}_{p+1}, \dots,
\slp{F}_q$. The algorithm uses the following recurrent formula to calculate
$\varphi(p,q)$:
$$\varphi(p, q) = \begin{cases}
0 &\text{if } p=q, \\
\min_{r = p}^q(\varphi(p, r) + \varphi(r+1, q) +&\\
|\log(|f_{i+p}|+\dots+|f_{i+r}|) -& \\
\log(|f_{i+r+1}|+\dots+|f_{i+q}|)|) &\mbox{otherwise}.
\end{cases}$$
The formula is correct since the concatenation of $\slp{F}_p$ and $\slp{F}_q$
generates at most. $2\cdot|h(\slp{F}_p) - h(\slp{F}_q)|$ rotations  In the
formula we omit the constant factor 2 since, otherwise, all values of $\varphi$
will be proportional to the factor.

The algorithm calculates the values of $\varphi(p, q)$ using a $\varphi$-table.
The $\varphi$-table is a $((k-1) \times (k-1))$-table that stores information
about the value of $\varphi(p,q)$ and about the optimal value of index $r$. The
cell in the $p$-th row and the $q$-th column where $p \leq q$ always contains
zero values. Initially the algorithm fills out cells such that $q - p = 1$,
next it fills out cells such that $q - p = 2$, etc. Thus, the algorithm does
not recompute recursively the values of $\varphi(p, r)$ and $\varphi(r+1, q)$
since they already exist in the $\varphi$-table.

Every single value of $\varphi(p, q)$ can be calculated using $O(k)$ time (the
pseudo-code of the corresponding procedure is presented in Figure 2). Thus, the
algorithm fills out the $\varphi$-table using $O(k^3)$ time and $O(k^2)$ space.

\begin{figure}[th]
\ComputingFiValue
\end{figure}

Notice that $\varphi(1, k)$ is equal to the optimal number of rotations of the
grammar tree in the worst case (i.e.\ in the case when each concatenation of
$\slp{F}_p$ and $\slp{F}_q$ generates $O(|h(\slp{F}_p) - h(\slp{F}_q)|)$
rotations). Since the $\varphi$-table contains information about the optimal
value of index $r$ in every cell, the algorithm can determine an optimal order
of concatenations for $\slp{F}_{i+1}, \dots, \slp{F}_{i+k}$. Using the optimal
order, the algorithm constructs an SLP $\slp{F}$ that derives $F_{i+1},
F_{i+2}, \dots, F_{i+l}$. Finally the algorithm concatenates $\slp{S}$ and
$\slp{F}$ and sets $\slp{S}$ to be equal to $\slp{S} \cdot \slp{F}$.

\begin{thm}
Let $f_1, f_2, \cdots, f_k$ be the LZ-factorization of a text $w$. The above
algorithm constructs an SLP for $w$ of size $O(k\log n)$.
\end{thm}
\proof{} The LZ-factorization can be divided into independent groups of
factors. Let $\ell$ be the number of the groups and let $k_1, k_2, \dots,
k_\ell$ be the sizes of the corresponding groups. The number of new rules
generated during concatenation of all factors in the $i$-th group is upper
bounded by $2\cdot(\log (|f_{k_i}| + |f_{k_i + 1}| + \dots +
|f_{k_{i+1}-1}|)\cdot(k_{i+1} - k_i))$. So the number of new rules generated
during a single step of the algorithm is upper bounded by
$2\cdot(\log(|f_{k_1}| + |f_{k_1 + 1}| + \dots + |f_{k_i -1}|) + (k_{i+1} - k_i
- 1)\cdot(\log (|f_{k_i}| + \dots + |f_{k_{i+1}-1}|))$. Finally the number of
rotations is upper bounded by $2\cdot(k_{i+1} - k_i)\cdot \log n$.

Totally the algorithm generates at most $O(k\log n)$ new rules.

Since concatenation of groups of factors is optimal in the worst case, it is
difficult to determine the precise time complexity of the improved algorithm as
a function of input size. Notice that if the size of every group of factors is
equal to 1, then we have the algorithm from \cite{SLPConstruction}. Otherwise
the new algorithm generates less rotations than the standard algorithm but the
new algorithm spends some extra time for calculating order of concatenation. So
in the next section we propose a practical comparison between the two
algorithms.

\section{Practical results}

\subsection{Setup of experiments}
Obviously, the nature of input strings highly affects compression
time and compression ratio. In this paper we consider three types of strings:

\begin{itemize}
  \item DNA sequences (downloaded from DNA Data Bank of Japan, \url{http://www.ddbj.nig.ac.jp/});
  \item Fibonacci strings;
  \item random strings over four letters alphabet.
\end{itemize}

These types of strings have been chosen for the following reasons. Fibonacci
strings are one of the best inputs to the SLP Construction problem. So we can
estimate potential of SLPs as compression model. Random strings are considered
to be incompressible and, potentially, they are the worst input for the SLP
Construction problem. DNA sequences are real data used in practice.

Let us introduce algorithms that have been tested. Our test suite contains a
classic version of the Lempel-Ziv algorithm proposed in \cite{LZ77}. We use the
following brief notation for it: \textbf{lz77}. The size of scanning window is
equal to 32Kb. The reader may think that 32Kb is too short but we choose this
value to study how the window size influences compression ratio.

Notice that the size of scanning window affects the length of factors and the
total size of a LZ-factorization. So if we want to compress longer factors,
then we need to spend more space. In other words, the size of scanning window
directly affects the size of compressed presentation. In the same time there is
no direct relation between the structure of an SLP and the size of the text
derived from it. Therefore using shorter LZ-factorizations leads to
constructing more compact SLPs. From this argument it follows that there is a
reason to consider a version of the Lempel-Ziv algorithm with an infinite
scanning window. We use the following brief notation for it: \textbf{lz77inf}.

Additionally we test yet another classic algorithm from the Lempel-Ziv family:
the Lempel-Ziv-Welch algorithm (briefly \textbf{lzw}). Finally our test suite
contains the classic algorithm for SLP Construction problem (briefly
\textbf{SLPclassic}) proposed in \cite{SLPConstruction} and the algorithm
presented in Section~3 (briefly \textbf{SLPnew}).

For convenience, we accept the following common notation for the tested
algorithms in all following graphs:

\algorithmNotations

We estimate the performance of a compression algorithm in terms of compression
ratio and execution time. For SLP compression algorithms we additionally
calculate the number of rotations. We calculate compression ratio as the ratio
between the size of compressed presentation and the size of input string. We
measure compression ratio in percents. For example, the formula for SLP
compression ratio looks like $\frac{|\slp{S}|}{|S|} \cdot 100$. Since ratio
between serialized SLP rule representation and serialized LZ-factor
representation is greater than 1 (but constant) we may encounter the following
situation: SLP compression ratio is less than LZ compression ratio but the file
containing an SLP is larger than the file containing the corresponding
LZ-dictionary. We should note that the situation changes with input growing.

All algorithms have been implemented using Java SE and tested on Intel Core
i-5(661), 4Gb RAM (Microsoft Windows 7 x64 OS).

\DNARotations

\subsection{SLP construction results}

In this subsection we discuss performance of both SLP construction algorithms.

As expected, both algorithms work infinitely fast on Fibonacci strings and
construct extremely compact representations. For example, on the 36-th
Fibonacci word of size 36.9Mb both algorithms work within 1ms and build SLPs of
size 100. This fact shows that there is a class of strings whereon the SLP
compression model is very efficient.

\DNASpeedTestInMemory

A diagram at Figure 3 shows how the suggested optimization of concatenation
affects the number of rotations of AVL trees on DNA sequences.

\DNASpeedTestInFile

To illustrate how the optimization affects  construction time we present two
tests. In the first one, the algorithms have stored all SLPs being constructed
in operational memory, while in the second one, the SLPs have been stored in an
external file so that every rotation of an AVL-tree forces I/O operations with
the file. We expect that there is no big difference between the algorithms in
the first setting since the new algorithm spends extra time when computing the
most efficient order of concatenation. But what happens when costs of accessing
AVL-trees are greater than costs of computing the order? This situation
naturally appear when the size of input is huge and we cannot store all the
trees in operational memory. So at figures 4 and 5 presented the execution time statistic for both ways of SLPs
storing.

\RandomRotations

\RandomSpeedTestInMemory

\RandomSpeedTestInFile

At Figures 6, 7 and 8 presented the results of testing both algorithms on random strings. 

The Figures 3-8 shows that the new algorithm is more stable since it is on
average as fast as the classic algorithm in operational memory and on average
two times faster when a file system is used.

\subsection{Compression ratio results}

Here we present a comparative analysis of compression ratios for DNA sequences
and random strings:

Figures 9 and 10 shows that compression ratio provided by the new algorithm is
close to compression ratio provided by the classic algorithm. In case of DNA
sequences the new algorithm generates many rotations of small AVL trees and as
a result it constructs a wider SLP than the classic algorithm. Also tests shows
that SLP compression ratio directly depends on the size of LZ-factorization
(fluctuations in the results obtained by both classic and new algorithms are
similar to fluctuations in the results obtained by \textbf{lz77inf} and the
size of compressed presentation obtained by \textbf{lz77inf} is equal to the
LZ-factorization size).

\DNACompression

\section{Conclusion}

In this paper we present an improved algorithm for SLP construction. The
algorithm is similar to Rytter's algorithm proposed in \cite{SLPConstruction}
but it seems to be more efficient on large inputs. However, there is an open
issue related to the new algorithm: how to manage groups of factors optimally?
For example, suppose that we already have built an SLP $\slp{L}$ of height $h$
and suppose that the next group of factors generates an SLP $\slp{R}$ of height
$2h$. It looks inefficient to construct the whole $\slp{R}$ and then
concatenate it with $\slp{L}$. A possible solution is to let the allowed size
of the group gradually grow as the algorithm progresses on the input string.

In the paper we also present practical tests of SLPs as compression model. As a
result we should conclude that the existing ways of SLP construction are quite
slow. On the other hand, the tests confirm that compression ratio provided by
SLPs is close to compression ratio provided by the family of Lempel-Ziv
algorithms (and may be even closer than we initially expected).

\RandomCompression

\section*{Acknowledgment}

The authors would like to thank \emph{Mikhail V. Volkov} for his encouragement
and critical notes and \emph{Anna Kozlova} for her help in gathering
statistics. The authors would like to thank anonymous referees for their reviews.

\begin{thebibliography}{1}
\bibitem{LCSubstring}
A.~Ishino, A. Shinohara, T.~Nakamura, W.~Matsubara, S.~Inenaga, and
K.~Hashimoto.
\newblock Computing longest common substring and all palindromes from
  compressed strings.
\newblock In {\em SOFSEM}, volume 4910 of {\em Lecture Notes in Computer
  Science}, pages 364--375. Springer, 2008.

\bibitem{PM_and_HD}
Y.~Lifshits.
\newblock Processing compressed texts: A tractability border.
\newblock In {\em CPM}, volume 4580 of {\em Lecture Notes in Computer Science},
  pages 228--240. Springer, 2007.

\bibitem{SLPConstruction}
W.~Rytter.
\newblock Application of Lempel-Ziv factorization to the approximation of
  grammar-based compression.
\newblock {\em Theor. Comput. Sci.}, 302(1-3):211--222, 2003.

\bibitem{collages}
Y.~Shibata, M. Takeda, A.~Shinohara, T.~Kida, T.~Matsumoto, and S.~Arikawa.
\newblock Collage system: a unifying framework for compressed pattern matching.
\newblock {\em Theor. Comput. Sci.}, 1(298):253--272, 2003.

\bibitem{antidictionaries}
A.~Shinohara, Y.~Shibata, M.~Takeda, and S.~Arikawa.
\newblock Pattern matching in text compressed by using antidictionaries.
\newblock In {\em CPM}, volume 1645 of {\em Lecture Notes in Computer Science},
  pages 37--49. Springer, 1999.

\bibitem{LCS_P}
A. Tiskin.
\newblock Faster subsequence recognition in compressed strings.
\newblock {\em CoRR}, abs/0707.3407, 2007.

\bibitem{LZW}
T.~A. Welch.
\newblock A technique for high-performance data compression.
\newblock {\em IEEE Computer}, 17(6):8--19, 1984.

\bibitem{LZ77}
J.~Ziv and A.~Lempel.
\newblock A universal algorithm for sequential data compression.
\newblock {\em IEEE Transactions on Information Theory}, 23(3):337--343, 1977.

\bibitem{LZ78}
J.~Ziv and A.~Lempel.
\newblock Compression of individual sequences via variable-rate coding.
\newblock {\em IEEE Transactions on Information Theory}, 24(5):530--536, 1978.
\end{thebibliography}
\end{document}
