\documentclass{beamer}

\mode<presentation>
{
\usetheme{Darmstadt}
\setbeamercovered{transparent}
}

\usepackage[english]{babel}
\usepackage[latin1]{inputenc}
\usepackage{mathptmx}
\usepackage[scaled=.90]{helvet}
\usepackage{courier}
\usepackage[T1]{fontenc}
\usepackage{pgf}
\usepackage{tikz}
\usepackage{common}

\pgfdeclareimage[width=8cm]{Raccoon}{zzz}

\title{Straight-line Programs: A Practical Test}

\author{Ivan Burmistrov \inst{\mbox{\textdagger}} and Lesha Khvorost
\inst{\mbox{\textdagger}}}

\institute
{
\inst{\mbox{\textdagger}}
Department of Mathematics and Mechanics\\
Ural State University}

\date{22th June CCP 2011}
\subject{Talks}

\AtBeginSubsection[]
{
\begin{frame}<beamer>
\frametitle{Outline}
\tableofcontents[currentsection,currentsubsection]
\end{frame}
}

\begin{document}

\begin{frame}
\titlepage
\end{frame}

\section{Introduction}

\begin{frame}
\frametitle{}

\begin{block}{Observation}
With growing input size of classic string algorithms (\textbf{Pattern
Matching}, \textbf{Longest Common Substring}, etc.) changes the algorithms that
efficiently process the input.
\end{block}

\pause
There exist the following approaches:

\begin{itemize}
  \item Processing the input using a file system (\textbf{I/O efficient
  algorithms});
  \pause
  \item Store and process the input as compressed  
  representation (\textbf{algorithms on compressed representations});
\end{itemize}
\end{frame}

\section{SLP Background}

\begin{frame}
\frametitle{Straight-line Program Definition}

\begin{definition}
	A \textbf{straight-line program} (SLP) $\slp{S}$ is a sequence of assignments
	of the form:
	
	\begin{center}
		$\slp{S}_1 = expr_1,\ \slp{S}_2 = expr_2, \dots, \slp{S}_n = expr_n,$
	\end{center}
	where $\slp{S}_i$ are \textbf{rules} and $expr_i$ are expressions of the form:
	\begin{itemize}
		\item $expr_i \in \Sigma$ (\textbf{terminal} rules), or
		\item $expr_i = \slp{S}_\ell\cdot \slp{S}_r \ (\ell, r < i)$
		(\textbf{nonterminal} rules).
	\end{itemize}
\end{definition}

\end{frame}

\begin{frame}
\frametitle{The Fibonacci Word SLP}

\begin{example}[SLP for <<$abaababaabaab$>>]
	\begin{center}
		\FibonacciWordSLP
	\end{center}
\end{example}

\end{frame}

\begin{frame}
\frametitle{Features of SLPs}

\begin{block}{Positive features}
	\begin{itemize}
	  \item<1-> Well-structured data representation;
	  \item<2-> Polynomial relation between the size of SLP for a given text and
	  the size of LZ77-dictionary for the same text;
	  \item<3-> There are classic string problems that have polynomial time
	  algorithms depends on size of SLPs (\textbf{Pattern Matching},
	  \textbf{Longest Common Substring}, \textbf{Computing All Squares}).
	\end{itemize}
\end{block}

\begin{alertblock}{Negative features}<4->
	\begin{itemize}
	  \item<4-> Constants hidden in big-\emph{O} notation for algorithms on SLPs
	  are often very big;
	  \item<5-> Polynomial relation between the size of an SLP for a given text and
	  the size of the LZ77-dictionary for the same text doesn't yet guarantee that
	  SLPs provide good compression ratio in practice;
	\end{itemize}
\end{alertblock}

\end{frame}

\begin{frame}
\frametitle{}

\begin{alertblock}{Main Question}
Whether or not there exist SLP-based compression models suitable
to practical usage?
\end{alertblock}

\pause

\begin{itemize}
  \item<2-> How difficult is it to compress data to an SLP-representation?
  \item<3-> How large compression ratio do SLPs provide as compared to classic
  algorithms used in practice?
\end{itemize}
\end{frame}

\section{Classic SLP Construction Approach}

\begin{frame}
\frametitle{SLP Construction Problem}

\begin{block}{SLP Construction Problem}
\textsc{Input:} A text $S$;

\textsc{Output:} An SLP $\slp{S}$ that derives text $S$;
\end{block}

\pause

\begin{theorem}
The problem of constructing a minimal size grammar generating a given text is
NP-hard.
\end{theorem}
\end{frame}

\begin{frame}
\frametitle{Idea of Factorization}

\begin{definition}
A \textbf{factorization} of a text $S$ is a set of strings $F_1, F_2,
\dots F_k$ such that $S = F_1 \cdot F_2 \cdot \mbox{\dots} \cdot F_k$.
\end{definition}

\pause

\begin{definition}
The \textbf{LZ-factorization} of a text $S$ is set of strings: $F_1 \cdot F_2
\cdot \mbox{\dots} \cdot F_k$, where

\begin{itemize}
  \item $F_1 = S[0]$;
  \item $F_i$ is the longest prefix of $S\substr{|F_1\cdot \mbox{\dots} \cdot
  F_{i-1}|}{|S|}$ which occurs earlier in the text \textbf{or} $S[|F_1\cdot
\mbox{\dots} \cdot F_{i-1}|]$ in case this prefix is empty.
\end{itemize}
\end{definition}

\pause

\begin{example}[Factorizations of <<$abaababaabaab$>>]
\begin{itemize}
  \item $a \cdot b \cdot a \cdot a \cdot b \cdot a \cdot b \cdot a \cdot a
  \cdot b \cdot a \cdot a \cdot b$;
  \item $a \cdot b \cdot a \cdot aba \cdot baaba \cdot ab$;
\end{itemize}
\end{example}
\end{frame}

\begin{frame}
\frametitle{Rytter's Approach}

\begin{block}{SLP Construction Problem}
\textsc{Input:} A text $S$ and its LZ-factorization $F_1, F_2, \dots, F_k$;

\textsc{Output:} An SLP $\slp{S}$ that derives text $S$;
\end{block}

\pause

\begin{example}
	\begin{center}
		\RytterExample
	\end{center}
\end{example}

\end{frame}

\begin{frame}
\frametitle{SLP Construction Example}

\begin{theorem}[Rytter]
Given a string $S$ of length $n$ and its LZ-factorization of length $k$, one
can construct an SLP for $S$ of size $O(k \cdot \log n)$ in time $O(k \cdot \log
n)$.
\end{theorem}

\pause

\begin{block}{Features}
	\begin{itemize}
	  \item Sequential factors processing;
	  \item Use AVL-trees as SLPs data representation;
	\end{itemize}
\end{block}
\end{frame}

\section{Modern Approach}
\begin{frame}
\frametitle{Bottleneck of The Algorithm}

\begin{alertblock}{Sequential factors processing}<1->
At every iteration the algorithm concatenates potentially huge and small
AVL-trees.
\end{alertblock}

\begin{example}<2->
	\begin{center}
		\BottleneckExample
	\end{center}
\end{example}

\end{frame}

\begin{frame}
\frametitle{Modern Approach}

\begin{block}{Key idea}
To optimize number of rotations by coupling factors into groups and calculating
optimal concatenation order.
\end{block}

\pause

\begin{theorem}
Given a string $S$ of length $n$ and its LZ-factorization of length $k$, one
can construct an SLP for $S$ of size $O(k \cdot \log n)$.
\end{theorem}

\pause

\begin{lemma}
Given sequence of SLPs of length $k$, one can compute optimal order of
concatenation using $O(k^3)$ time.
\end{lemma}
\end{frame}

\section{Results}

\begin{frame}
\frametitle{Types of Input Data}

\begin{itemize}
  \item Fibonacci words;
  \item DNA sequences from DNA Data Bank of Japan (\url{http://www.ddbj.nig.ac.jp/});
  \item Random texts on 4-letters alphabet; 
\end{itemize}

\end{frame}

\begin{frame}
\frametitle{Number of Rotations Results}

\begin{center}
	\DNARotations
\end{center}
\end{frame}

\begin{frame}
\frametitle{Speed Results in Operational Memory}

\begin{center}
	\DNASpeedTestInMemory
\end{center}
\end{frame}

\begin{frame}
\frametitle{Speed Results on File System}

\begin{center}
	\DNASpeedTestInFile
\end{center}
\end{frame}

\begin{frame}
\frametitle{Compression Results}

\begin{center}
	\DNACompression
\end{center}
\end{frame}

\section*{Conclusion}

\begin{frame}
\frametitle{Conclusion}

\begin{itemize}
  \item We present modification of the algorithm for SLP Construction Problem;
  \item We compare performance of both algorithms on practise;
  \item We compare compression ratio provided by classic encoding algorithms and
  by two SLP-encoding algorithms;
\end{itemize}

\pause

\begin{alertblock}{Open Problem}
How to optimally choose size of a group of factors?
\end{alertblock}
\end{frame}

\begin{frame}
\frametitle{Questions?}

\begin{center}
	\begin{tikzpicture}
		\draw (0,0) node {\pgfuseimage{Raccoon}};
	\end{tikzpicture}
\end{center}
%TODO cool picture
\end{frame}

\end{document}
