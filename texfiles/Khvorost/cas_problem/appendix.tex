\begin{appendix}
\section{Examples}

\subsection{Leech word square-freeness}
Let us consider the SLP $\slp{L}$ that derives substring of Leech square free word: 

\begin{center}
$\slpterm{L}{1}{a}, \slpterm{L}{2}{b}, \slpterm{L}{3}{c}, \slpnonterm{L}{4}{2}{3}, \slpnonterm{L}{5}{2}{1},
\slpnonterm{L}{6}{1}{4}$

$\slpnonterm{L}{7}{3}{4}, \slpnonterm{L}{8}{6}{5}, \slpnonterm{L}{9}{8}{4}, \slpnonterm{L}{10}{8}{7},
\slpnonterm{L}{11}{10}{9}$
\end{center}

Parse tree for $\slp{L}$ presented at figure 4.

\begin{figure}[th]
\LeechSLP
\end{figure}

Let us check square freeness of $\slp{L}$. Since $|L| = 15$ then the algorithm check the following segments of root
length: [1, 1], [2, 3], [4, 7]. Notice that rules of $\slp{L}$ already ordered by length of derived text.

\begin{itemize}
  \item \textbf{checking square freeness of squares that root length equals to 1}
  
  Firstly the algorithm skip terminal rules. Next it consecutively look for squares exactly in other rules. For
  instance, let us consider how it checks $\slpnonterm{L}{10}{8}{7}$. The algorithm find $L_8[5] = a$ and $L_7[1] = c$
  using $\slp{L}_{10}$. Since $a \neq c$ it moves to $\slp{L}_{11}$.
  
  It is clear that $\slp{L}_4, \dots, \slp{L}_{11}$ have no squares of length 2 around its cut positions.
  \item \textbf{checking square freeness of squares that root length belong to [2, 3]}
  
  The algorithm skip all rules that have length less 4 (i.e. $\slp{L}_1, \dots, \slp{L}_7$). For instance, let us
  consider how it checks $\slpnonterm{L}{8}{6}{5}$. Since $|L_8| = 5$ and block length is equal to 1 then the algorithm
  construct an SLP $\slp{L}_8'$ that derives text \emph{\$~\$~\$~\$~\$~a~b~c~b~a~\$~\$~\$~\$~\$~\$}, where \emph{\$} is
  special symbol not from $\Sigma$. At figure 5 presented partition of $L_8'$ into blocks of length 1.
  
  \LeechWordPartitionSimple
  
  Remind that the algorithm necessary to check blocks $B_5, \dots, B_{12}$. It no need to check $B_5$ since it contains
  \emph{\$}. Also it no need to check $B_9$ and $B_{10}$ since the search area is out of $L_8$. For instance, let us
  consider how it checks $B_7$. The algorithm build $\slp{L}_8'\substr{6}{7}$ that derives $B_7$ and  
  $\slp{L}_8'\substr{8}{11}$ that derives search area $B_9 \cdot B_{10}$. Next it runs pattern matching algorithm on
  $\slp{L}_8'\substr{6}{7}$, $\slp{L}_8'\substr{8}{11}$ and obtain occurrence of $B_7$ at position 9 of $L_8'$. Finally
  it runs \textbf{SubsExt} problem with the following parameters: $\slp{L}_8'$ and positions 6, 7, 8, 9. The algorithm
  obtains $\ell_{ex} = r_{ex} = 0$. Since $\ell_{ex} + r_{ex} = 0 = a - (k - 1) \cdot 2^{i-2} = 8 - (9 - 1) \cdot 1$ then
  algorithm moves to $L_9$.

  \item \textbf{checking square freeness of squares that root length belongs to [4, 7]}
  
  The algorithm skip all rules that have length less 8 (i.e. $\slp{L}_1, \dots, \slp{L}_9$). For instance, let us
  consider how it checks $\slpnonterm{L}{11}{10}{9}$. Firstly it construct an SLP $\slp{L}_{11}'$ that derives $L$
  surrounded with \emph{\$} and $|L_{11}'| = 32$. At figure 6 presented partition of $L_{11}'$ into blocks of length
  2. 
  
  \LeechWordPartitionComplex

  Next the algorithm find no squares for $B_5, \dots, B_8$ since pattern matching algorithm returns empty set of
  results. Next the algorithm find occurrence of $B_9$ at position 22 of $L_{11}'$ and run \textbf{SubsExt}
  problem with the following parameters: $\slp{L}_{11}'$ and positions 16, 18, 20, 22. Since $\ell_{ex} = r_{ex} = 0$ the
  algorithm moves to $B_{10}$. Finally the algorithm find no squares for $B_{10}, B_{11}, B_{12}$ since correspond
  search areas contains \emph{\$}.
\end{itemize}

\subsection{Construction PS-table for $\slp{F}_7$}

Let us consider how the algorithm construct PS-table for $\slp{F}_7$ that derives text \emph{a~b~a~a~b~a~b~a~a~b~a~a~b}.
The PS-table size is equal to $(\lfloor\log |F_7|\rfloor+1)\times (|\slp{F}_7|+1) = 4 \times 8$. 

\begin{itemize}
  \item \textbf{looking for squares that root length equals to 1}
  
  Firstly the algorithm mark cells with $\varnothing$ for rules with length less than 2. Next it consecutively look for
  squares exactly in other rules. PS(1, 3) = $\varnothing$ since $F_1[1] = a \neq b = F_2[1]$. Analogously PS(1, 4) =
  PS(1, 6) = $\varnothing$. PS(1, 5) = \{1, 3, 3\} since $F_4[3] = a = F_3[1]$. PS(1, 7) = \{1, 8, 8\} since $F_6[8] = a
  = F_5[1]$. After first step PS-table has the following view: 
  
  \begin{figure}[h]
	  {\footnotesize\noindent
		\begin{tabular}{|c|c|c|c|c|c|c|c|} \hline
	 	& $\slpterm{F}{1}{a}$ & $\slpterm{F}{2}{b}$ & $\slpnonterm{F}{3}{1}{2}$ & $\slpnonterm{F}{4}{3}{1}$ &
	 	$\slpnonterm{F}{5}{4}{3}$ & $\slpnonterm{F}{6}{5}{4}$ & $\slpnonterm{F}{7}{6}{5}$ \\ \hline
	
	 	[1, 1] & $\varnothing$ & $\varnothing$ & $\varnothing$ & $\varnothing$ & \{1, 3, 3\} & $\varnothing$ & \{1, 8, 8\} \\
	 	\hline
	
	 	[2, 3] & & & & & & & \\ \hline
	
	 	[4, 7] & & & & & & & \\ \hline
		\end{tabular}
	  }
  \end{figure}
  
  \item \textbf{looking for squares that root length belongs to [2, 3]}
  
  The algorithm mark cells with $\varnothing$ for rules with length less than 4. For instance, let us consider
  how the algorithm fill PS(2, 6). The algorithm construct SLP $\slp{F}_6'$ that derives text
  \emph{\$~\$~\$~a~b~a~a~b~a~b~a~\$~\$~\$~\$~\$} of length 16. At figure 7 presented partition of $F_6'$ into blocks.
  
  \FibonacciWordPartition
  
  $B_5$: using \textbf{PM} problem the algorithm obtain occurrence of $B_5$ at position 8 of $F_6'$ on  
  $\slp{F}_{6}'\substr{5}{6}, \slp{F}_{6}'\substr{7}{9}$; the substring extending algorithm obtain $\ell_{ex} = r_{ex} = 1$
  on $\slp{F}_{6}'$ with parameters 5, 6, 8, 9; so family of repetitions \{3, 7, 7\} was found;
  
  $B_6$: using \textbf{PM} problem the algorithm obtain occurrence of $B_6$ at position 9 of $F_6'$ on  
  $\slp{F}_{6}'\substr{6}{7}, \slp{F}_{6}'\substr{8}{10}$; the substring extending algorithm obtain $\ell_{ex} = 2, r_{ex}
  = 0$ on $\slp{F}_{6}'$ with parameters 6, 7, 9, 10; so family of repetitions \{3, 7, 7\} was found;
  
  $B_7$: using \textbf{PM} problem the algorithm obtain occurrence of $B_7$ at position 9 of $F_6'$ on  
  $\slp{F}_{6}'\substr{7}{8}, \slp{F}_{6}'\substr{9}{11}$; the substring extending algorithm obtain $\ell_{ex} = 0, r_{ex}
  = 1$ on $\slp{F}_{6}'$ with parameters 7, 8, 9, 10; so no family of repetitions was found;
  
  $B_8$: using \textbf{PM} problem the algorithm obtain occurrence of $B_8$ at position 10 of $F_6'$ on  
  $\slp{F}_{6}'\substr{8}{9}, \slp{F}_{6}'\substr{10}{12}$; the substring extending algorithm obtain $\ell_{ex} = r_{ex}
  = 1$ on $\slp{F}_{6}'$ with parameters 8, 9, 10, 11; so family of repetitions \{2, 8, 9\} was found;
  
  $B_9$: using \textbf{PM} problem the algorithm obtain occurrence of $B_9$ at position 11 of $F_6'$ on  
  $\slp{F}_{6}'\substr{9}{10}, \slp{F}_{6}'\substr{11}{13}$; the substring extending algorithm obtain $\ell_{ex} = 1,
  r_{ex} = 0$ on $\slp{F}_{6}'$ with parameters 9, 10, 11, 12; so family of repetitions \{2, 9, 9\} was found;
  
  The algorithm skip blocks $B_{10}, \dots, B_{12}$ since the search area consist of \emph{\$}. After merging families 
  of repetitions the algorithm have the following result: \{3, 7, 7\}, \{2, 8, 9\}. Finally the algorithm check purity
  of families and shift them form $F_{6}'$ to $F_6$. 
  
  After second step PS-table has the following view:
  
  \begin{figure}[h]
	  {\footnotesize\noindent
		\begin{tabular}{|c|c|c|c|c|c|c|c|} \hline
	 	& $\slpterm{F}{1}{a}$ & $\slpterm{F}{2}{b}$ & $\slpnonterm{F}{3}{1}{2}$ & $\slpnonterm{F}{4}{3}{1}$ &
	 	$\slpnonterm{F}{5}{4}{3}$ & $\slpnonterm{F}{6}{5}{4}$ & $\slpnonterm{F}{7}{6}{5}$ \\ \hline
	
	 	[1, 1] & $\varnothing$ & $\varnothing$ & $\varnothing$ & $\varnothing$ & \{1, 3, 3\} & $\varnothing$ & \{1, 8, 8\} \\
	 	\hline
	
	 	[2, 3] & $\varnothing$ & $\varnothing$ & $\varnothing$ & $\varnothing$ & $\varnothing$ & \{3, 4, 4\}, \{2, 5, 6\} &
	 	\{3, 10, 10\} \\ 	\hline
	
	 	[4, 7] & & & & & & & \\ \hline
		\end{tabular}
	  }
  \end{figure}
  
  \item \textbf{looking for squares that root length belongs to [4, 7]}
  
  The algorithm mark cells with $\varnothing$ for rules with length less than 8. The algorithm construct SLP
  $\slp{F}_7'$ that surrounds with \emph{\$} and has length 32. At figure 8 presented partition of $F_7'$ into
  blocks. Let us consider how the algorithm fill PS(3, 7). It skips $B_5$ since $B_5$ contains \emph{\$}. 
  
   \FibonacciWordPartitionComplex
  
  $B_6$: using \textbf{PM} problem the algorithm obtain occurrence of $B_6$ at position 15; the substring extending algorithm
  obtain $\ell_{ex} = 1, r_{ex} = 3$ on $\slp{F}_{7}'$ with parameters 10, 12, 14, 18; so family of repetitions \{5, 14,
  15\} was found;
  
  $B_7$: using \textbf{PM} problem the algorithm obtain occurrence of $B_7$ at position 17; the substring extending algorithm
  obtain $\ell_{ex} = 3, r_{ex} = 1$ on $\slp{F}_{7}'$ with parameters 12, 14, 16, 20; so family of repetitions \{5, 14,
  15\} was found;
  
  $B_8$: using \textbf{PM} problem the algorithm obtain occurrence of $B_8$ at position 20; the substring extending algorithm
  obtain $\ell_{ex} = r_{ex} = 0$ on $\slp{F}_{7}'$ with parameters 14, 16, 18, 22; so there are no families of
  repetitions;
  
  $B_9$: using \textbf{PM} problem the algorithm obtain no occurrence of $B_9$; so there are no families of repetitions;
  
  The algorithm skips $B_{10}$ and $B_{11}$ since search areas consist of \emph{\$}. It skips $B_{12}$ since $B_{12}$
  contains \emph{\$}. After merging families of repetitions the algorithm have the following result: \{5, 14, 15\}. 
  Finally the algorithm check purity of families and shift them form $F_{7}'$ to $F_7$.
  
  Finally PS-table has the following view:
  
  \begin{figure}[h]
	  {\footnotesize\noindent
		\begin{tabular}{|c|c|c|c|c|c|c|c|} \hline
	 	& $\slpterm{F}{1}{a}$ & $\slpterm{F}{2}{b}$ & $\slpnonterm{F}{3}{1}{2}$ & $\slpnonterm{F}{4}{3}{1}$ &
	 	$\slpnonterm{F}{5}{4}{3}$ & $\slpnonterm{F}{6}{5}{4}$ & $\slpnonterm{F}{7}{6}{5}$ \\ \hline
	
	 	[1, 1] & $\varnothing$ & $\varnothing$ & $\varnothing$ & $\varnothing$ & \{1, 3, 3\} & $\varnothing$ & \{1, 8, 8\} \\
	 	\hline
	
	 	[2, 3] & $\varnothing$ & $\varnothing$ & $\varnothing$ & $\varnothing$ & $\varnothing$ & \{3, 4, 4\}, \{2, 5, 6\} &
	 	\{3, 10, 10\} \\ \hline
	
	 	[4, 7] & $\varnothing$ & $\varnothing$ & $\varnothing$ & $\varnothing$ & $\varnothing$ & $\varnothing$ & \{5, 5, 6\}
	 	\\ \hline
		\end{tabular}
	  }
  \end{figure}
\end{itemize}

\subsection{Text with complex family of pure squares}
Let us consider an SLP $\slp{E} = \slp{E}_l \cdot \slp{E}_r$ such that $\slp{E}_l$ derives text $(a~b~a~b~a)^6~a~b~a$
and $\slp{E}_r$ derives text $b~a~a~b~a~(a~b~a~b~a)^7$. So $\slp{E}$ derives text $(a~b~a~b~a)^7~a~b~a~(a~b~a~b~a)^7$
and $|E| = 73$. Let us consider how the algorithm looking for pure squares for rule $\slp{E}$ and segment [16, 31]. The 
algorithm construct SLP $\slp{E}'$ that surrounds with \emph{\$} and has length 128. The partition of $E'$ into blocks
presented at figure 9. 

\ComplexExamplePartititon

Let us consider how the algorithm process block $B_8$. It find occurrence of $B_8$ in $B_9 \cdot B_{10}$ at positions
$\prog{72}{2}{5}$. Since the algorithm find more than one occurrence it extends periodicity to calculate parameters
$\alpha_L, \alpha_R, \gamma_L, \gamma_R$. So $\alpha_L = \infty, \alpha_R = 69, \gamma_L = 66, \gamma_R = \infty$.
According to lemma !!! the algorithm find complex family of pure squares associated with each of root length \{18, 23\} 
and centred at positions \{66, 67, 68, 69\}. Next we write every root from the family exactly: 

\begin{itemize}
  \item \textbf{roots of length 18}

  $a~b~a(a~b~a~b~a)^3$, $b~a(a~b~a~b~a)^3a$, $a(a~b~a~b~a)^3a~b$, $(a~b~a~b~a)^3a~b~a$;
  \item \textbf{roots of length 23}
  
  $a~b~a(a~b~a~b~a)^4$, $b~a(a~b~a~b~a)^4a$, $a(a~b~a~b~a)^4a~b$, $(a~b~a~b~a)^4a~b~a$;
\end{itemize}

\end{appendix}