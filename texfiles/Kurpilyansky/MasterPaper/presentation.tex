\documentclass[10pt,pdf,hyperref={unicode}]{beamer}

\mode<presentation>
{
\usetheme{Darmstadt}
\setbeamercovered{transparent}
} 

\usepackage[utf8]{inputenc}
\usepackage[english,russian]{babel}
\usepackage{amssymb}
\usepackage{multicol}
\usepackage{url}
\usepackage{pgf}
\usepackage{tikz}
\usepackage{scalefnt}
\usepackage{color}

\usepackage{graphs2}
\usepackage{trees}

\renewcommand{\drawLzw}[2]{
}

\newcommand{\slpAlgorithmsNotations}{
	\begin{tikzpicture}
		\drawRytter{-2}{1.0};
		\drawModernFirstRytter{-2}{0.5};
		\drawModernSecondRytter{-2}{0.0};
		\drawCartesian{3}{1.0};
		\drawAvlConcurrentFour{3}{0.5};
		\drawLcaOnline{3}{0};
        
		\node [right] at (-2,1.0) {\scriptsize {\sf -- Риттер;}};
		\node [right] at (-2,0.5) {\scriptsize {\sf -- Риттер с эвристикой;}};
		\node [right] at (-2,0.0) {\scriptsize {\sf -- Риттер с эвристикой {\color{red} New!};}};

		\node [right] at (3,1.0) {\scriptsize {\sf -- рандомизированные ПП;}};
		\node [right] at (3,0.5) {\scriptsize {\sf -- многопоточный алгоритм {\color{red} New!};}};
		\node [right] at (3,0.0) {\scriptsize {\sf -- LCA-online.}};
	\end{tikzpicture}
}

\newcommand{\slpAndLzAlgorithmsNotations}{
\scalebox{0.8}{
	\begin{tikzpicture}
		\drawLz{-2}{1.5};
		\drawRytter{-2}{1.0};
		\drawModernFirstRytter{-2}{0.5};
		\drawModernSecondRytter{-2}{0.0};

		\drawLzInf{3}{1.5};
		\drawCartesian{3}{1.0};
		\drawAvlConcurrentFour{3}{0.5};
		\drawLcaOnline{3}{0};
        
		\node [right] at (-2,1.5) {\scriptsize {\sf -- LZ с окном сжатия 32КБ;}};
		\node [right] at (-2,1.0) {\scriptsize {\sf -- Риттер;}};
		\node [right] at (-2,0.5) {\scriptsize {\sf -- Риттер с эвристикой;}};
		\node [right] at (-2,0.0) {\scriptsize {\sf -- Риттер с эвристикой {\color{red} New!};}};

		\node [right] at (3,1.5) {\scriptsize {\sf -- LZ;}};
		\node [right] at (3,1.0) {\scriptsize {\sf -- рандомизированные ПП;}};
		\node [right] at (3,0.5) {\scriptsize {\sf -- многопоточный алгоритм {\color{red} New!};}};
		\node [right] at (3,0.0) {\scriptsize {\sf -- LCA-online.}};
	\end{tikzpicture}
}
}

\title{Классификация алгоритмов построения прямолинейных программ}

\author{Евгений Курпилянский}

\date{11 июня 2014 года}

\begin{document}

\begin{frame}
\titlepage
\end{frame}

\section{Введение}

\begin{frame}{Способы сжатия}

\begin{block}{}
	Существует различные способы сжатия данных, например:
	\begin{itemize}
		\item прямолинейные программы;
		\item антисловари;
		\item коллаж-системы и др.
	\end{itemize}
\end{block}

\pause

\begin{block}{}
	Если сжатое представление хорошо {\bf структурировано}, то существуют алгоритмы, способные
	решать классические задачи\newline {\bf без распаковки} данных.
\end{block}

\pause

\begin{exampleblock}{Примеры таких задач}
	\begin{itemize}
		\item поиск подстроки в строке;
		\item наибольшая общая подстрока;
		\item и др.
	\end{itemize}
\end{exampleblock}

\end{frame}

\begin{frame}

	\begin{block}{Вопросы}
		\begin{enumerate}
			\item Умеем ли мы эффективно строить маленькие ПП?
			\pause
			\item Насколько эффективно решение классических задач в терминах ПП?
		\end{enumerate}
	\end{block}

	\pause

		Данная работа посвящена первому вопросу.
\end{frame}

\section{Прямолинейные программы}

\begin{frame}
\frametitle{Определение прямолинейной программы}

\begin{block}{Определение}
\textbf{Прямолинейная программа} (ПП) строки $S$ ~-- это
контекстно-свободная грамматика в нормальной форме Хомского, выводящая в точности одно слово $S$.
\end{block}

\pause

\begin{exampleblock}{Пример}
Рассмотрим ПП $X$, выводящую строку $\mbox{<<}abaababaabaab\mbox{>>}$.

\begin{center}
$X_1 \to a$\\
$X_2 \to b$\\
$X_3 \to X_1 \cdot X_2$\\
$X_4 \to X_3 \cdot X_1$\\
$X_5 \to X_4 \cdot X_3$\\
$X_6 \to X_5 \cdot X_4$\\
$X_7 \to X_6 \cdot X_5$
\end{center}
\end{exampleblock}

\end{frame}

\begin{frame}
\begin{exampleblock}{Пример}
Графическое изображение ПП:

\picFibonacciSLP
\end{exampleblock}
\end{frame}

\section{Построение ПП}

\begin{frame}
\frametitle{Как строить ПП?}
\begin{block}{Утверждение.}
Задача построения минимальной ПП, выводящей заданную строку $S$~-- NP-трудная.
\end{block}

\pause

\begin{center}
$\Downarrow$
\end{center}

\begin{block}{}
Для построения ПП требуется использовать приближенные алгоритмы.
\end{block}
\end{frame}

\begin{frame}
\begin{block}{Определение}
\textbf{Факторизация} строки $S$~-- это набор строк $w_1$, $w_2$, \ldots, $w_k$ такой,
что $S = w_1 \cdot w_2 \cdot \ldots \cdot w_k$.
\end{block}
\pause
\begin{block}{Определение}
\textbf{$LZ$-факторизация} строки $S$ ~--- это факторизация $S~=~w_1~\cdot~w_2~\cdots~w_k$ такая, что
для любого $j \in 1..k$
\begin{itemize}
	\item $w_j$ состоит из одной буквы, не встречающейся в $w_1 \cdot w_2 \cdots w_{j-1}$; или
	\item $w_j$ ~--- наибольший префикс $w_j \cdot w_{j+1} \cdots w_k$, встречающийся в $w_1 \cdot w_2 \cdots w_{j-1}$.
\end{itemize}
\end{block}

\begin{exampleblock}{Факторизации строки <<$abaababaabaab$>>}
\begin{itemize}
  \item $a \cdot b \cdot a \cdot a \cdot b \cdot a \cdot b \cdot a \cdot a
  \cdot b \cdot a \cdot a \cdot b$;
  \item $a \cdot b \cdot a \cdot aba \cdot baaba \cdot ab$ (LZ-факторизация);
\end{itemize}
\end{exampleblock}
\end{frame}

\begin{frame}
\begin{block}{Нижняя оценка (Риттер, 2001)}
	Размер минимальной ПП, выводящей данный текст, не меньше размера LZ-факторизации этого текста.
\end{block}

\pause

\begin{block}{Постановка задачи}
\textsc{Вход:} Строка $T$.

\textsc{Выход:} ПП, выводящая строку $T$.
\end{block}

\pause

\begin{block}{Постановка задачи 2}
\textsc{Вход:} Строка $T$ {\bf и ее LZ-факторизация} $F_1$, $F_2$, \ldots, $F_k$.

\textsc{Выход:} ПП, выводящая строку $T$.
\end{block}

\end{frame}

\begin{frame}

\only<1-6>{
\begin{block}{Постановка задачи}
\textsc{Вход:} Строка $T$ и ее $LZ$-факторизация $F_1$, $F_2$, \ldots, $F_k$.

\textsc{Выход:} ПП, выводящая строку $T$.
\end{block}
}

\begin{block}{Алгоритмы построения, основанные на сбалансированных бинарных деревьях:}
	\begin{itemize}
		\pause
		\item Алгоритм Риттера {\bf (W.~Rytter, 2001)}
		\pause
		\item Эвристическая оптимизация {\bf (И.~Бурмистров, А.~Хворост, 2011)}
		\pause
		\item Небольшое обобщение эвристики {\color{red} New!}
		\pause
		\item Алгоритм построения рандомизированных ПП {\bf (Е.~Курпилянский, 2012)}
		\pause
		\item Многопоточный алгоритм {\color{red} New!}
	\end{itemize}
\end{block}

\pause

\only<7->{
\begin{block}{Характеристики}
	\pause
	Все эти алгоритмы:
	\begin{itemize}
		\item строят $O(\log n)$-приближение к минимальной;
		\item требуют на вход LZ-факторизацию текста.
	\end{itemize}
\end{block}
}

\end{frame}

\begin{frame}
	\only<1-3>{
		\begin{block}{Постановка задачи}
			\textsc{Вход:} Строка $T$.

			\textsc{Выход:} ПП, выводящая строку $T$.
		\end{block}
	}

	\begin{block}{Алгоритмы построения ПП, не требующие построения LZ-факторизации.}

		\pause

		\only<1-3>{Самый лучший алгоритм по качеству сжатия имеет оценку $O(\log^2 n)$.}

		\pause

		LCA-online алгоритм {\bf (S.~Maruyama, H.~Sakamoto, M.~Takeda, 2012)}
	\end{block}

	\pause

	\only<4->{
		\begin{block}{Характеристики}
			\pause
			Данный алгоритм:
			\begin{itemize}
				\item строит $O(\log^2 n)$-приближение к минимальной;
				\item требует на вход только сам текст;
				\pause
				\item работает online.
			\end{itemize}
		\end{block}
	}
\end{frame}

\section{Практические результаты}
\begin{frame}
\frametitle{Практические результаты}

Алгоритмы были протестированы на:
\begin{itemize}
	\item последовательности строк с большой LZ-факторизацией ($\Omega(\frac{n}{\log n})$);
	\item случайных строках над четырехбуквенным алфавитом;
	\item ДНК, взятых с сайта \url{http://www.ddbj.nig.ac.jp/}.	
\end{itemize}

\end{frame}

\begin{frame}
\frametitle{Скорость работы на строках ДНК}

\begin{figure}[!ht]
	\begin{center}
        \begin{picture}(260,165)(0,0)
			\scalebox{0.65}{
				\buildingSlpTimesInMemoryForDna
			}
		\end{picture}
	\end{center}
	\begin{center}
		\slpAlgorithmsNotations
	\end{center}
\end{figure}
\end{frame}

\begin{frame}
\frametitle{Скорость работы на случайных строках}

\begin{figure}[!ht]
	\begin{center}
        \begin{picture}(260,165)(0,0)
			\scalebox{0.65}{
				\buildingSlpTimesInMemoryForRandom
			}
		\end{picture}
	\end{center}
	\begin{center}
		\slpAlgorithmsNotations
	\end{center}
\end{figure}
\end{frame}

\begin{frame}
\frametitle{Отношение размеров представлений на строках ДНК}

\begin{figure}[!ht]
	\begin{center}
        \begin{picture}(260,165)(0,0)
			\scalebox{0.65}{
				\compressionItemSizeForDna
			}
		\end{picture}
	\end{center}
	\begin{center}
		\slpAndLzAlgorithmsNotations
	\end{center}
\end{figure}
\end{frame}

\begin{frame}
\frametitle{Отношение размеров представлений на случайных стр.}

\begin{figure}[!ht]
	\begin{center}
        \begin{picture}(260,165)(0,0)
			\scalebox{0.65}{
				\compressionItemSizeForRandom
			}
		\end{picture}
	\end{center}
	\begin{center}
		\slpAndLzAlgorithmsNotations
	\end{center}
\end{figure}
\end{frame}

\begin{frame}
\frametitle{Коэффициент сжатия на строках ДНК}

\begin{figure}[!ht]
	\begin{center}
        \begin{picture}(260,165)(0,0)
			\scalebox{0.65}{
				\compressionByteSizeForDna
			}
		\end{picture}
	\end{center}
	\begin{center}
		\slpAndLzAlgorithmsNotations
	\end{center}
\end{figure}
\end{frame}

\begin{frame}
\frametitle{Коэффициент сжатия на случайных строках}

\begin{figure}[!ht]
	\begin{center}
        \begin{picture}(260,165)(0,0)
			\scalebox{0.65}{
				\compressionByteSizeForRandom
			}
		\end{picture}
	\end{center}
	\begin{center}
		\slpAndLzAlgorithmsNotations
	\end{center}
\end{figure}
\end{frame}

\begin{frame}
\frametitle{Количеств групп факторов в многопоточном алгоритме на строках ДНК}

\begin{figure}[!ht]
	\begin{center}
        \begin{picture}(260,180)(0,0)
			\scalebox{0.75}{
				\concurrentIterationCountForDna
			}
		\end{picture}
	\end{center}
\end{figure}
\end{frame}

\begin{frame}
\frametitle{Количеств групп факторов в многопоточном алгоритме на случайных строках}

\begin{figure}[!ht]
	\begin{center}
        \begin{picture}(260,180)(0,0)
			\scalebox{0.75}{
				\concurrentIterationCountForRandom
			}
		\end{picture}
	\end{center}
\end{figure}
\end{frame}


\section{Заключение}
\begin{frame}
\frametitle{Результаты и дальнейшие планы}

\begin{block}{Результаты}
	\begin{itemize}
	  \item Обобщение эвристики оптимизации порядка конкатенаций.
	  \pause
	  \item Многопоточный алгоритм построения ПП.
	  \pause
	  \item Практические результаты сравнения алгоритмов по двум параметрам: скорость работы и качество сжатия.
	  \pause
	  \item LCA-online лучше всех?
	\end{itemize}
	Исходные коды алгоритмов можно посмотреть здесь: \url{http://code.google.com/p/overclocking}
\end{block}

\pause

\begin{block}{Планы}
	\begin{itemize}
	  \item Исследовать оценку качества сжатия LCA-online алгоритма.
	  \pause
	  \item Доделать реализацию многопоточного алгоритма.
	\end{itemize}
\end{block}

\end{frame}

\end{document}
