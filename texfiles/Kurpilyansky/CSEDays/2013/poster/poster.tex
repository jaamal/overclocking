% --------------------------------------------------------------------------- %
% Poster for the CSEDays 2013.                                                %
% --------------------------------------------------------------------------- %
% Created with Brian Amberg's LaTeX Poster Template. Please refer for the     %
% attached README.md file for the details how to compile with `pdflatex`.     %
% --------------------------------------------------------------------------- %
% $LastChangedDate:: 2011-09-11 10:57:12 +0200 (V, 11 szept. 2011)          $ %
% $LastChangedRevision:: 128                                                $ %
% $LastChangedBy:: rlegendi                                                 $ %
% $Id:: poster.tex 128 2011-09-11 08:57:12Z rlegendi                        $ %
% --------------------------------------------------------------------------- %
\documentclass[a1paper,portrait,fontscale=0.45]{baposter}

\usepackage{relsize}		% For \smaller
\usepackage{url}			% For \url
\usepackage{epstopdf}	% Included EPS files automatically converted to PDF to include with pdflatex
\usepackage{mathrsfs}
\usepackage{amsmath}
\DeclareSymbolFont{rsfscript}{OMS}{rsfs}{m}{n}
\DeclareSymbolFontAlphabet{\mathrsfs}{rsfscript}
\DeclareMathOperator{\Syn}{Syn}

\usepackage{tikz}
\usepackage{graphs}
\usepackage{amsmath,amssymb,amsthm}

%%% Global Settings %%%%%%%%%%%%%%%%%%%%%%%%%%%%%%%%%%%%%%%%%%%%%%%%%%%%%%%%%%%

\graphicspath{{pix/}}	% Root directory of the pictures
\tracingstats=2			% Enabled LaTeX logging with conditionals
\newtheorem{thm}{Theorem}

%%% Color Definitions %%%%%%%%%%%%%%%%%%%%%%%%%%%%%%%%%%%%%%%%%%%%%%%%%%%%%%%%%

\definecolor{bordercol}{RGB}{40,40,40}
\definecolor{headercol1}{RGB}{186,215,230}
\definecolor{headercol2}{RGB}{80,80,80}
\definecolor{headerfontcol}{RGB}{0,0,0}
\definecolor{boxcolor}{RGB}{186,215,230}

%%%%%%%%%%%%%%%%%%%%%%%%%%%%%%%%%%%%%%%%%%%%%%%%%%%%%%%%%%%%%%%%%%%%%%%%%%%%%%%%
%%% Utility functions %%%%%%%%%%%%%%%%%%%%%%%%%%%%%%%%%%%%%%%%%%%%%%%%%%%%%%%%%%

%%% Save space in lists. Use this after the opening of the list %%%%%%%%%%%%%%%%
\newcommand{\compresslist}{
	\setlength{\itemsep}{1pt}
	\setlength{\parskip}{0pt}
	\setlength{\parsep}{0pt}
}

\newcommand{\slp}[1]{\mathbb{#1}}

%%%%%%%%%%%%%%%%%%%%%%%%%%%%%%%%%%%%%%%%%%%%%%%%%%%%%%%%%%%%%%%%%%%%%%%%%%%%%%%
%%% Document Start %%%%%%%%%%%%%%%%%%%%%%%%%%%%%%%%%%%%%%%%%%%%%%%%%%%%%%%%%%%%
%%%%%%%%%%%%%%%%%%%%%%%%%%%%%%%%%%%%%%%%%%%%%%%%%%%%%%%%%%%%%%%%%%%%%%%%%%%%%%%

\begin{document}
\typeout{Poster rendering started}

%%% Setting Background Image %%%%%%%%%%%%%%%%%%%%%%%%%%%%%%%%%%%%%%%%%%%%%%%%%%
\background{
	\begin{tikzpicture}[remember picture,overlay]%
	\draw (current page.north west)+(-2em,2em) node[anchor=north west]
	{\includegraphics[height=1.1\textheight]{background}};
	\end{tikzpicture}
}

%%% General Poster Settings %%%%%%%%%%%%%%%%%%%%%%%%%%%%%%%%%%%%%%%%%%%%%%%%%%%
%%%%%% Eye Catcher, Title, Authors and University Images %%%%%%%%%%%%%%%%%%%%%%
\begin{poster}{
	grid=false,
	% Option is left on true though the eyecatcher is not used. The reason is
	% that we have a bit nicer looking title and author formatting in the headercol
	% this way
	%eyecatcher=false,
	borderColor=bordercol,
	headerColorOne=headercol1,
	headerColorTwo=headercol2,
	headerFontColor=headerfontcol,
	% Only simple background color used, no shading, so boxColorTwo isn't necessary
	boxColorOne=boxcolor,
	headershape=roundedright,
	headerfont=\sf\bf,
	textborder=rectangle,
	background=user,
	headerborder=open,
    boxshade=plain
}
%%% Eye Cacther %%%%%%%%%%%%%%%%%%%%%%%%%%%%%%%%%%%%%%%%%%%%%%%%%%%%%%%%%%%%%%%
{
	Eye Catcher, empty if option eyecatcher=false - unused
}
%%% Title %%%%%%%%%%%%%%%%%%%%%%%%%%%%%%%%%%%%%%%%%%%%%%%%%%%%%%%%%%%%%%%%%%%%%
{\bf
	Straight-line Programs:\\
    A Short Overview
}
%%% Authors %%%%%%%%%%%%%%%%%%%%%%%%%%%%%%%%%%%%%%%%%%%%%%%%%%%%%%%%%%%%%%%%%%%
{
	\vspace{1em} Eugene B. Kurpilyansky\\
	{\smaller dembelz@yandex.ru}
}
%%% Logo %%%%%%%%%%%%%%%%%%%%%%%%%%%%%%%%%%%%%%%%%%%%%%%%%%%%%%%%%%%%%%%%%%%%%%
{
% The logos are compressed a bit into a simple box to make them smaller on the result
% (Wasn't able to find any bigger of them.)
\setlength\fboxsep{0.5pt}
\setlength\fboxrule{0.5pt}
	\fbox{
		\begin{minipage}{15em}
			\begin{center}
				\includegraphics[scale=0.65]{zzz.jpg}
			\end{center}
		\end{minipage}
	}
}

\headerbox{Definitions}{name=definitions,column=0,row=0}{

Straight-line program (SLP) is a context-free grammar in Chomsky normal form which produces exactly one string over $\Sigma^+$.
	%\begin{figure}[hb]
    \begin{center}
        \begin{picture}(30,95)(75,0)
            \fibonacciwordslp
        \end{picture}
    \end{center}
    %\caption{An SLP that derives $abaababaabaab$}
	%\end{figure}
}

\headerbox{Main Facts}{name=main,column=0,below=definitions}{
\begin{thm}{\cite{SmallestCFG}}
There is no polynomial-time algorithm for the smallest grammar problem with approximation ratio less than
$\frac{8569}{8568}$ unless P = NP
\end{thm}

\begin{thm}{\cite{SLPConstruction}}
We can construct in $O(n \log |\Sigma|)$ time a $O(\log n)$-ratio approximation of a minimal grammar-based compression. Given LZ-factorization of length $k$ we can construct a correcponding grammar of size $O(k \log n)$ in time $O(k \log n)$.
\end{thm}
}

\headerbox{Complexity}{name=complexity,column=0,below=main}{

There is a class of string problems that can be solved in terms of SLPs. Let $\mathbb{T}$ be an SLP that derives a text and let $\mathbb{P}$ be an SLP that derives a pattern. 
\begin{center}
\begin{tabular}{|l|c|}
\hline
    \textbf{Problem} & \textbf{Time} \\ \hline
	Pattern matching & $O(|\mathbb{T}|^2|\mathbb{P}|)$ \\ \hline
	Counting all palindromes & $O(|\mathbb{T}|^4$) \\ \hline
	Longest common substring & $O(|\mathbb{T}|^4 \log |\mathbb{T}|)$ \\ \hline
	Computing all overlaps & $O(|\mathbb{T}|^4 \log |\mathbb{T}|)$ \\
\hline
\end{tabular}
\end{center}

There is a class of string problems that have no polynomial algorithm in terms of SLPs.
\begin{center}
\begin{tabular}{|l|c|}
\hline
    \textbf{Problem} & \textbf{Complexity} \\ \hline
	Hamming distance & \#P-complete \\ \hline
	Embedding & $\theta^p_2$-hard \\ \hline
	Longest common subsequence & NP-hard \\
\hline
\end{tabular}
\end{center}
}

\headerbox{References}{name=references,column=0,below=complexity}{
\smaller													% Make the whole text smaller
\vspace{-0.4em} 										% Save some space at the beginning
\bibliographystyle{plain}							% Use plain style
\renewcommand{\section}[2]{\vskip 0.05em}		% Omit "References" title
\begin{thebibliography}{1}							% Simple bibliography with widest label of 1
\itemsep=-0.01em										% Save space between the separation
\setlength{\baselineskip}{0.4em}					% Save space with longer lines

\bibitem{Springer}
\textsl{I.\,Burmistrov, A.\,Kozlova, E.\,Kurpilyansky, A.\,Khvorost}, Straight-line programs: a practical test (extended abstract),
Journal of Mathematical Sciences, 192 (2013), 282--294.

\bibitem{SmallestCFG}
\textsl{M.\,Charikar, E.\,Lehman,  D.\,Liu, R.\,Panigrahy, M.\,Prabhakaran, A.\,Sahai, A.\,Shelat}, The smallest grammar problem, IEEE
Trans. Information Theory, 51 (2005), 2554--2576.

\bibitem{SLPConstruction}
\textsl{W.\,Rytter}, Application of {L}empel-{Z}iv factorization to the approximation of grammar-based compression,
Theor. Comput. Sci., 302 (2003), 211--222.

\end{thebibliography}
}

\headerbox{A Prctical Test}{name=comparisonAlgorithms,span=2,column=1,row=0}{
	We are investigating different approaches that improve Rytter's algorithm:
	\begin{itemize}
		\item To minimize number of AVL-rotations using dynamic programming \cite{Springer}
		\item To spend less time on balancing of nodes using Cartesian trees instead of AVL trees \cite{Springer}
		\item To construct AVL trees in parallel. 
	\end{itemize}
	Algorithms notations:
	\begin{center}
		\begin{picture}(50,60)(70,0)
	    		\algorithmNotations
	    \end{picture}
    \end{center}
    
    AVL rotations optimization statistics:
	\begin{center}
		\begin{picture}(50,200)(70,0)
        	\DNARotations
		\end{picture}
	\end{center}
	SLPs construction time statistics on DNAs: 
	\begin{center}
    	\begin{picture}(50,170)(70,0)
        	\DNATimeStats
		\end{picture}
	\end{center}
	Compressions ratio statistics on DNAs:
    \begin{center}
        \begin{picture}(50,240)(70,0)
        	\CompressionRatioOnDNAs
        \end{picture}
    \end{center}
}

\end{poster}
\end{document}
